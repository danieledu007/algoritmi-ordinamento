\documentclass{article}
\usepackage{media9}
\usepackage{graphicx}
\usepackage{animate}

\author{Guillen, Youssef}
\title{Algoritmi di ordinamento e i loro casi di uso}
\begin{document}

\maketitle
\section*{Introduzione:}
\section*{bubble sort:}
Il Bubble Sort è un algoritmo di ordinamento il cui funzionamento consiste nel confrontare coppie di elementi contenuti in una lista di dati
e invertirli se uno degli elementi si trova nell'ordine sbagliato. Questo processo si ripete 
fino a quando la lista è ordinata. Questo algoritmo è più adatto per liste di piccole dimensioni poiché ha una complessità 
temporale di \(O(n^2)\) nel caso medio e nel peggiore. Viene spesso utilizzato per insegnare i concetti fondamentali di ordinamento.
\begin{figure}[ht!]
    \centering
    \animategraphics[loop,autoplay]{10}{C:/Users/daniel/Desktop/algoritmi-ordinamento/relazione/media/Bubblesort/Bubblesort2-}{0}{207}
\end{figure}

\section*{Selection Sort:}
ssssss
\begin{figure}[ht!]
    \centering
    \animategraphics[loop,autoplay]{10}{C:/Users/daniel/Desktop/algoritmi-ordinamento/relazione/media/Selectionsort/Selectionsort2-}{0}{560}
\end{figure}
\section*{Mergesort}
sssssss
\begin{figure}[ht!]
    \centering
    \animategraphics[loop,autoplay,scale=0.44]{10}{C:/Users/daniel/Desktop/algoritmi-ordinamento/relazione/media/MergeSort/Mergesort-}{0}{340}
\end{figure}

abbiamo
\end{document}